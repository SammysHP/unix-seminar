\documentclass{beamer}

%--- includes ------------------------------------------------------------------
\usepackage[T1]{fontenc}
\usepackage[utf8]{inputenc}
\usepackage[ngerman]{babel}
\usepackage{helvet}

\usepackage{etex}

\usepackage{xspace}
\xspaceaddexceptions{"=}
\usepackage{microtype}
\usepackage{fixltx2e}

% \usepackage{amsmath}
% \usepackage{amssymb}
% \usepackage{mathtools}

% \usepackage{graphicx}
% \usepackage{xcolor}
\usepackage{tikz}
\usetikzlibrary{arrows,automata,calc,chains,trees,positioning,scopes,decorations.pathmorphing,shapes,backgrounds,tikzmark}

% \usepackage{multicol}
% \usepackage{tabu}
% \usepackage{multirow}
% \usepackage{booktabs}

\usepackage{listings}
\usepackage[noend]{algpseudocode}
\usepackage{fp}

% \usepackage{hyperref}

%--- colors and layout ---------------------------------------------------------
\usetheme{default}
\usecolortheme{rose}

\definecolor{uniblue}{rgb}{0,0.31,0.62}
\colorlet{maincolor}{uniblue}
\setbeamercolor{structure}{fg=maincolor}
\setbeamercolor{block title alerted}{fg=red!90!black}

\setbeamertemplate{navigation symbols}{}
\setbeamertemplate{blocks}[rounded]
\setbeamertemplate{footline}{\raisebox{5pt}{\makebox[\paperwidth]{\color{lightgray}\scriptsize\hyperlink{TableOfContents}{\insertframenumber}}}}
\setbeamertemplate{section page}{
    \centering
    \begin{beamercolorbox}[sep=12pt,center]{part title}
      \usebeamerfont{section title}\insertsection\par
    \end{beamercolorbox}
}

\setbeamerfont{title}{series=\bfseries,parent=structure}
\setbeamerfont{subtitle}{size=\footnotesize,series=\normalfont,parent=structure}

%--- code listings -------------------------------------------------------------
\lstdefinestyle{basestyle}{
    language=C,
    tabsize=4,
    frame=none,
    xleftmargin=15pt,
    framexleftmargin=15pt,
    numbersep=4pt,
    aboveskip=0pt,
    belowskip=0pt,
    numbers=left,
    firstnumber=1,
    stepnumber=1,
    basewidth=0.45em,
    showstringspaces=false,
    basicstyle=\ttfamily\fontsize{7pt}{8pt}\selectfont,
    numberstyle=\color{lightgray}\fontsize{5pt}{6pt}\selectfont,
    identifierstyle=\color{black},
    commentstyle=\color{gray},
    % keywordstyle=\bfseries\color{red!70!black},
    % stringstyle=\color{green!50!black},
    keywordstyle=\bfseries\color{black},
    stringstyle=\color{black},
    % breaklines=true, % Turn off in preview, this makes it fucking slow
    postbreak=\raisebox{0ex}[0ex][0ex]{\ensuremath{\color{red}\hookrightarrow\space}},
    literate={Ö}{{\"O}}1 {Ä}{{\"A}}1 {Ü}{{\"U}}1 {ß}{{\ss}}1 {ü}{{\"u}}1 {ä}{{\"a}}1 {ö}{{\"o}}1,
}

\lstdefinestyle{nonumbers}{
    style=basestyle,
    xleftmargin=0pt,
    framexleftmargin=0pt,
    numbersep=0pt,
    numbers=none,
    % backgroundcolor=\color{black!10}
}

\lstset{
    style=basestyle
}

\makeatletter\newcommand{\insertcode}[5]{
    % USAGE:
    %   \insertcode{filename}{linecount}{offset}{from}{to}
    % filename:     Name of the source file
    % linecount:    Number of lines in file
    % offset:       Number of first line
    % from:         Start with this line (relative to file)
    % to:           End with this line (relative to file)
    \vfill
    \makebox[\linewidth][c]{
        \begin{tikzpicture}[remember picture, inner sep=0pt]
            \path[use as bounding box] (-0.7,3.9) rectangle (11.5,-3.9);

            \node (overview) {
                \resizebox{8.5mm}{!}{\begin{minipage}{10cm}
                    \lstinputlisting[style=nonumbers, name=overview]{#1}
                \end{minipage}}
            };

            \node (code) [right=8mm of overview] {
                \begin{minipage}{10cm}
                    \FPeval{\result}{clip(#3+#4-1)}
                    \lstinputlisting[firstline=#4, lastline=#5, firstnumber=\result, name=code]{#1}
                \end{minipage}
            };

            \node [below=1mm of overview] {\tiny\itshape\filename@parse{#1}\filename@base.\filename@ext};

            \begin{pgfonlayer}{background}
                \coordinate (overview-start) at ($(overview.north west)!{(#4-1)/#2}!(overview.south west)+(-1pt,0)$);
                \coordinate (overview-end) at ($(overview.north west)!#5/#2!(overview.south west)$);
                % \path [fill=maincolor!10, rounded corners=1pt] (overview-start) rectangle ($(overview.south east |- overview-end)+(-4pt,0)$);
                \path [fill=maincolor!10, rounded corners=1pt] (overview-start) rectangle ($(overview.south east |- overview-end)+(1pt,0)$);
                % \path [shade, left color=maincolor!10] (code.north west) to [out=180, in=0, looseness=0.2] ($(overview.south east |- overview-start)+(-5pt,0)$) -- ($(overview.south east |- overview-end)+(-5pt,0)$) to [out=0, in=180, looseness=0.2] (code.south west);
            \end{pgfonlayer}
        \end{tikzpicture}
    }
    \vfill
}\makeatother

\algrenewcommand{\alglinenumber}[1]{\color{lightgray}\scriptsize #1\ }
\makeatletter\renewcommand{\ALG@beginalgorithmic}{\small}\makeatother

%--- metadata ------------------------------------------------------------------
\title{Kapitel 20 -- Dateisysteme}
\subtitle{aus \emph{Lions' Commentary on UNIX 6th Edition}}
\author{Sven Greiner\\\footnotesize\ttfamily sven@sammyshp.de}
\date{27. Januar 2015}

%--- text replacements ---------------------------------------------------------
\newcommand{\zB}{z.\,B.~}
\newcommand{\ZB}{Z.\,B.~}
\newcommand{\dphp}{d.\,h.~}
\newcommand{\Dphp}{D.\,h.~}

\begin{document}

%===============================================================================


\begin{frame}[plain]
    \titlepage
\end{frame}


%-------------------------------------------------------------------------------


\begin{frame}[plain]{Inhalt}
    \label{TableOfContents}
    \vfill
    \begin{columns}[t]
        \begin{column}{.5\textwidth}
            \tableofcontents[sections={1}]
        \end{column}
        \begin{column}{.5\textwidth}
            \tableofcontents[sections={2}]
        \end{column}
    \end{columns}
    \vfill
\end{frame}


%-------------------------------------------------------------------------------


\section{Einführung}

\begin{frame}[plain]
    \sectionpage
\end{frame}


%-------------------------------------------------------------------------------


\subsection{Motivation}

\begin{frame}{Motivation}
    TODO
\end{frame}


%-------------------------------------------------------------------------------


\subsection{Begriffsklärung}

\begin{frame}{Begriffsklärung}
    TODO
\end{frame}


%-------------------------------------------------------------------------------


\subsection{Der Superblock}

\begin{frame}{Der Superblock}
    \begin{itemize}
        \item Auf jedem Datenträger in Block \#1 gespeichert
        \item In Buffer zwischengespeichert
        \item Wird regelmäßig auf Datenträger geschrieben
    \end{itemize}

    \vfill

    \lstinputlisting[firstline=12, lastline=27, firstnumber=5561]{source/filsys.h}
\end{frame}


%-------------------------------------------------------------------------------


\subsection{Die mount Tabelle}

\begin{frame}{Die \emph{mount} Tabelle}
    \begin{itemize}
        \item Für jedes eingebundene Dateisystem ein Eintrag mit wichtigen Zeigern

        \vfill

        \item systm.h:
            \vspace{5pt}
            \lstinputlisting[firstline=73, lastline=78, firstnumber=272]{source/systm.h}

        \vfill

        \item param.h:
            \vspace{5pt}
            \lstinputlisting[firstline=34, lastline=34, firstnumber=133]{source/param.h}

        \vfill
    \end{itemize}
\end{frame}


%-------------------------------------------------------------------------------


\section{Funktionen}

\begin{frame}[plain]
    \sectionpage
\end{frame}


%-------------------------------------------------------------------------------


\subsection{iinit()}

\begin{frame}{iinit()}
    \begin{itemize}
        \item TODO
        \item TODO
        \item TODO
        \item TODO
    \end{itemize}
\end{frame}

\begin{frame}{iinit()}
    \begin{algorithmic}[1]
        \State rootdev öffnen
        \State superblock in nicht zugewiesenen Buffer kopieren
        \State Zeiger in nullten Eintrag der mount Tabelle schreiben
        \State Locks explizit zurücksetzen
        \State als schreibbar markieren
        \State Systemzeit setzen
    \end{algorithmic}
\end{frame}

\begin{frame}{iinit()}
    \insertcode{source/alloc.c}{334}{6900}{23}{42}
\end{frame}


%-------------------------------------------------------------------------------


\subsection{smount()}

\begin{frame}{smount()}
    \begin{itemize}
        \item TODO
        \item TODO
        \item TODO
        \item TODO
    \end{itemize}
\end{frame}

\begin{frame}{smount()}
    \begin{algorithmic}[1]
        \State TODO
    \end{algorithmic}
\end{frame}

\begin{frame}{smount()}
    \insertcode{source/sys3.c}{198}{6000}{87}{103}
\end{frame}

\begin{frame}{smount()}
    \insertcode{source/sys3.c}{198}{6000}{104}{121}
\end{frame}

\begin{frame}{smount()}
    \insertcode{source/sys3.c}{198}{6000}{122}{139}
\end{frame}


%-------------------------------------------------------------------------------


\subsection{iget()}

\begin{frame}{iget()}
    \insertcode{source/iget.c}{243}{7250}{27}{32}
\end{frame}

\begin{frame}{iget()}
    \insertcode{source/iget.c}{243}{7250}{34}{59}
\end{frame}

\begin{frame}{iget()}
    \insertcode{source/iget.c}{243}{7250}{60}{85}
\end{frame}


%-------------------------------------------------------------------------------


\subsection{getfs()}

\begin{frame}{getfs()}
    \insertcode{source/alloc.c}{334}{6900}{268}{286}
\end{frame}


%-------------------------------------------------------------------------------


\subsection{update()}

\begin{frame}{update()}
    \insertcode{source/alloc.c}{334}{6900}{302}{332}
\end{frame}


%-------------------------------------------------------------------------------


\subsection{sumount()}

\begin{frame}{sumount()}
    \insertcode{source/sys3.c}{198}{6000}{145}{159}
\end{frame}

\begin{frame}{sumount()}
    \insertcode{source/sys3.c}{198}{6000}{161}{174}
\end{frame}


%-------------------------------------------------------------------------------


\subsection{alloc()}

\begin{frame}{alloc()}
    \insertcode{source/alloc.c}{334}{6900}{57}{85}
\end{frame}

\begin{frame}{alloc()}
    \insertcode{source/alloc.c}{334}{6900}{87}{92}
\end{frame}


%-------------------------------------------------------------------------------


\subsection{itrunc()}

\begin{frame}{itrunc()}
    \insertcode{source/iget.c}{243}{7250}{165}{174}
\end{frame}

\begin{frame}{itrunc()}
    \insertcode{source/iget.c}{243}{7250}{175}{195}
\end{frame}

\begin{frame}{itrunc()}
    \insertcode{source/iget.c}{243}{7250}{196}{200}
\end{frame}


%-------------------------------------------------------------------------------


\subsection{free()}

\begin{frame}{free()}
    \insertcode{source/alloc.c}{334}{6900}{101}{128}
\end{frame}


%-------------------------------------------------------------------------------


\subsection{iput()}

\begin{frame}{iput()}
    \insertcode{source/iget.c}{243}{7250}{95}{115}
\end{frame}


%-------------------------------------------------------------------------------


\subsection{ifree()}

\begin{frame}{ifree()}
    \insertcode{source/alloc.c}{334}{6900}{235}{246}
\end{frame}


%-------------------------------------------------------------------------------


\subsection{iupdat()}

\begin{frame}{iupdat()}
    \insertcode{source/iget.c}{243}{7250}{125}{153}
\end{frame}


\end{document}
