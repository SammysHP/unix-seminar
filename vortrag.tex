% \documentclass[aspectratio=169,10pt]{beamer}
\documentclass{beamer}

%--- includes ------------------------------------------------------------------
\usepackage[T1]{fontenc}
\usepackage[utf8]{inputenc}
\usepackage[ngerman]{babel}
\usepackage{helvet}
\usepackage{inconsolata}

\usepackage{etex}

\usepackage{xspace}
\xspaceaddexceptions{"=}
\usepackage{microtype}
\usepackage{fixltx2e}

\usepackage{amsmath}
\usepackage{amssymb}
\usepackage{mathtools}

\usepackage{graphicx}
\usepackage{xcolor}
\usepackage{tikz}
\usetikzlibrary{arrows,automata,calc,chains,trees,positioning,scopes,decorations.pathmorphing,decorations.pathreplacing,shapes,backgrounds,tikzmark,fadings}

\usepackage{multicol}
\usepackage{tabu}
\usepackage{multirow}
\usepackage{booktabs}

\usepackage{listings}
\usepackage[noend]{algpseudocode}
\usepackage{fp}

\usepackage{hyperref}

%--- colors and layout ---------------------------------------------------------
\usetheme{default}
\usecolortheme{rose}

\definecolor{uniblue}{rgb}{0,0.31,0.62}
\colorlet{maincolor}{uniblue}
\setbeamercolor{structure}{fg=maincolor}
\setbeamercolor{block title alerted}{fg=red!90!black}

\setbeamertemplate{navigation symbols}{}
\setbeamertemplate{blocks}[rounded]
\setbeamertemplate{footline}{\raisebox{5pt}{\makebox[\paperwidth]{\color{lightgray}\scriptsize\hyperlink{TableOfContents}{\insertframenumber}}}}
\setbeamertemplate{section page}{
    \centering
    \begin{beamercolorbox}[sep=12pt,center]{part title}
      \usebeamerfont{section title}\insertsection\par
    \end{beamercolorbox}
}

\setbeamerfont{title}{series=\bfseries,parent=structure}
\setbeamerfont{subtitle}{size=\footnotesize,series=\normalfont,parent=structure}

%--- code listings -------------------------------------------------------------
\lstdefinestyle{basestyle}{
    language=C,
    tabsize=4,
    frame=none,
    xleftmargin=15pt,
    framexleftmargin=15pt,
    numbersep=4pt,
    aboveskip=0pt,
    belowskip=0pt,
    numbers=left,
    firstnumber=1,
    stepnumber=1,
    basewidth=0.47em,
    showstringspaces=false,
    basicstyle=\ttfamily\fontsize{8pt}{9pt}\selectfont,
    numberstyle=\color{lightgray}\fontsize{5pt}{6pt}\selectfont,
    identifierstyle=,
    commentstyle=\color{gray},
    keywordstyle=,
    stringstyle=,
    % breaklines=true, % Turn off in preview, this makes it fucking slow
    postbreak=\raisebox{0ex}[0ex][0ex]{\ensuremath{\color{gray}\hookrightarrow\space}},
    showlines=true,
    literate={Ö}{{\"O}}1 {Ä}{{\"A}}1 {Ü}{{\"U}}1 {ß}{{\ss}}1 {ü}{{\"u}}1 {ä}{{\"a}}1 {ö}{{\"o}}1,
}

\lstdefinestyle{nonumbers}{
    style=basestyle,
    xleftmargin=0pt,
    framexleftmargin=0pt,
    numbersep=0pt,
    numbers=none,
    basicstyle=\ttfamily\fontsize{8pt}{8pt}\selectfont,
}

\lstset{
    style=basestyle
}

\tikzfading[name=code fading, bottom color=transparent!70, top color=transparent!100]
\newcommand*{\ContextLines}{5}
\newcommand*{\CodeWidth}{10.4cm}

\makeatletter\newcommand{\insertcode}[5]{
    % USAGE:
    %   \insertcode{filename}{linecount}{offset}{from}{to}
    % filename:     Name of the source file
    % linecount:    Number of lines in file
    % offset:       Number of first line
    % from:         Start with this line (relative to file)
    % to:           End with this line (relative to file)
    \vfill
    \makebox[\linewidth][c]{
        \begin{tikzpicture}[remember picture, inner sep=0pt]
            \path[use as bounding box] (-0.6,3.9) rectangle (11.7,-3.9);

            \node (overview) {
                \resizebox{8.5mm}{!}{\begin{minipage}{10cm}
                    \lstinputlisting[style=nonumbers]{#1}
                \end{minipage}}
            };

            \node (code) [right=8mm of overview] {
                \begin{minipage}{\CodeWidth}
                    \FPeval{\result}{clip(#3+#4-1)}
                    \lstinputlisting[firstline=#4, lastline=#5, firstnumber=\result]{#1}
                \end{minipage}
            };

            \node [scope fading=code fading] [above=0mm of code] {
                \begin{minipage}{\CodeWidth}
                    \FPeval{\ResultStart}{clip(#4-\ContextLines)}
                    \FPeval{\ResultEnd}{clip(#4-1)}
                    \FPeval{\ResultLine}{clip(#3+\ResultStart-1)}
                    \lstinputlisting[firstline=\ResultStart, lastline=\ResultEnd, firstnumber=\ResultLine]{#1}
                \end{minipage}
            };

            \node [scope fading=code fading, fading transform={rotate=180}] [below=0mm of code] {
                \begin{minipage}{\CodeWidth}
                    \FPeval{\ResultStart}{clip(#5+1)}
                    \FPeval{\ResultEnd}{clip(#5+\ContextLines)}
                    \FPeval{\ResultLine}{clip(#3+\ResultStart-1)}
                    \lstinputlisting[firstline=\ResultStart, lastline=\ResultEnd, firstnumber=\ResultLine]{#1}
                \end{minipage}
            };

            \node [below=1mm of overview] {\tiny\itshape\filename@parse{#1}\filename@base.\filename@ext};

            \begin{pgfonlayer}{background}
                \coordinate (overview-start) at ($(overview.north west)!{(#4-1)/#2}!(overview.south west)+(-1pt,0)$);
                \coordinate (overview-end) at ($(overview.north west)!#5/#2!(overview.south west)$);
                % \path [fill=maincolor!10, rounded corners=1pt] (overview-start) rectangle ($(overview.south east |- overview-end)+(-4pt,0)$);
                \path [fill=maincolor!10, rounded corners=1pt] (overview-start) rectangle ($(overview.south east |- overview-end)+(1pt,0)$);
                % \path [shade, left color=maincolor!10] (code.north west) to [out=180, in=0, looseness=0.2] ($(overview.south east |- overview-start)+(-5pt,0)$) -- ($(overview.south east |- overview-end)+(-5pt,0)$) to [out=0, in=180, looseness=0.2] (code.south west);
            \end{pgfonlayer}
        \end{tikzpicture}
    }
    \vfill
}\makeatother

\algrenewcommand{\alglinenumber}[1]{\color{lightgray}\scriptsize #1\ }
\makeatletter\renewcommand{\ALG@beginalgorithmic}{\small}\makeatother

%--- metadata ------------------------------------------------------------------
\title{Kapitel 20 -- Dateisysteme}
\subtitle{aus \emph{Lions' Commentary on UNIX 6th Edition}}
\author{Sven Greiner\\\footnotesize\ttfamily sven@sammyshp.de}
\date{27. Januar 2015}

%--- text replacements ---------------------------------------------------------
\newcommand{\zB}{z.\,B.~}
\newcommand{\ZB}{Z.\,B.~}
\newcommand{\dphp}{d.\,h.~}
\newcommand{\Dphp}{D.\,h.~}

\begin{document}

%===============================================================================


\begin{frame}[plain]
    \titlepage
\end{frame}


%-------------------------------------------------------------------------------


\begin{frame}[plain]{Inhalt}
    \label{TableOfContents}
    \vfill
    \begin{columns}[t]
        \begin{column}{.5\textwidth}
            \tableofcontents[sections={1}]
        \end{column}
        \begin{column}{.5\textwidth}
            \tableofcontents[sections={2}]
        \end{column}
    \end{columns}
    \vfill
\end{frame}


%-------------------------------------------------------------------------------


\section{Einführung}

\begin{frame}[plain]
    \sectionpage
\end{frame}


%-------------------------------------------------------------------------------


\subsection{Motivation}

\begin{frame}{Motivation}
    \begin{itemize}
        \item Zusammenführen von mehreren Dateisystemen
        \medskip
        \item Speichern und Lesen der Metadaten auf dem Datenträger
    \end{itemize}
\end{frame}


%-------------------------------------------------------------------------------


\subsection{Der Super Block}

\begin{frame}{Der Super Block}
    \begin{itemize}
        \item Auf jedem Datenträger in Block \#1 gespeichert
        \item In Buffer zwischengespeichert
        \item Wird regelmäßig auf Datenträger geschrieben
    \end{itemize}

    \medskip

    \lstinputlisting[firstline=12, lastline=27, firstnumber=5561]{source/filsys.h}
\end{frame}


%-------------------------------------------------------------------------------


\subsection{Disk Layout}

\begin{frame}<1>[label=disklayout]{Disk Layout}
    \centering
    \begin{tikzpicture}[start chain=going right, >=stealth', node distance=-.5pt]
        \tikzset{block/.style={minimum width=10mm, minimum height=15mm, draw}}

        \node [block, on chain, fill=maincolor!25] (boot) {0};
        \node [block, on chain, fill=maincolor!10] (superblock) {1};
        \node [block, on chain, fill=maincolor!25] (inode1) {2};
        \node [block, on chain, draw=none, minimum width=15mm] {\ldots};
        \node [block, on chain, fill=maincolor!25] (inode2) {};
        \node [block, on chain, fill=maincolor!10] (data1) {};
        \node [block, on chain, draw=none, minimum width=15mm] {\ldots};
        \node [block, on chain, fill=maincolor!10] (data2) {};

        \draw [decoration={brace,amplitude=0.7em},decorate,thick]
            ($(inode2.south east)+(-1pt,-2pt)$) -- ($(inode1.south west)+(1pt,-2pt)$)
            node [pos=0.5, below=3mm] {Inodes};
        \draw [decoration={brace,amplitude=0.7em},decorate,thick]
            ($(data2.south east)+(-1pt,-2pt)$) -- ($(data1.south west)+(1pt,-2pt)$)
            node [pos=0.5, below=3mm] {Daten};
        \draw [decoration={brace,amplitude=0.7em},decorate,thick]
            ($(superblock.north west)+(1pt,2pt)$) -- ($(superblock.north east)+(-1pt,2pt)$)
            node [pos=0.5, above=3mm] {512 B};
        \draw [decoration={brace,amplitude=0.7em},decorate,thick]
            ($(superblock.south east)+(-1pt,-2pt)$) -- ($(superblock.south west)+(1pt,-2pt)$)
            node [pos=0.5, below right=3mm and -5mm, align=center] {Super\\Block};
        \draw [decoration={brace,amplitude=0.7em},decorate,thick]
            ($(boot.south east)+(-1pt,-2pt)$) -- ($(boot.south west)+(1pt,-2pt)$)
            node [pos=0.5, below left=3mm and -5mm, align=center] {Boot\\Block};
    \end{tikzpicture}
\end{frame}


%-------------------------------------------------------------------------------


\subsection{Die mount Tabelle}

\begin{frame}{Die mount Tabelle}
    \begin{itemize}
        \item Für jedes eingebundene Dateisystem ein Eintrag mit wichtigen Zeigern

        \bigskip

        \item systm.h:
            \smallskip
            \lstinputlisting[firstline=73, lastline=78, firstnumber=272]{source/systm.h}

        \bigskip

        \item param.h:
            \smallskip
            \lstinputlisting[firstline=34, lastline=34, firstnumber=133]{source/param.h}
    \end{itemize}
\end{frame}


%-------------------------------------------------------------------------------


\section{Funktionen}

\begin{frame}[plain]
    \sectionpage
\end{frame}


%-------------------------------------------------------------------------------


\subsection{iinit()}

\begin{frame}{iinit()}
    \begin{itemize}
        \item (Frühes) Initialisieren der mount Tabelle für das Root Device
        \item \texttt{m\_inodp} wird nicht gesetzt und wird auch nie genutzt
        \item Setzen der Systemzeit
    \end{itemize}
\end{frame}

\begin{frame}{iinit()}
    \insertcode{source/alloc.c}{334}{6900}{23}{42}
\end{frame}


%-------------------------------------------------------------------------------


\subsection{smount()}

\begin{frame}{smount()}
    \begin{itemize}
        \item Mounten eines Dateisystems
        \item Keine Überprüfung, was gemountet wurde
        \item Keine Überprüfung des Hardware-Schreibschutzes
        \medskip
        \item Aufruf: \texttt{/etc/mount /dev/rk2 /rk2 [ro]}
    \end{itemize}
\end{frame}
\begin{frame}{smount()}
    \insertcode{source/sys3.c}{198}{6000}{87}{102}
\end{frame}

\begin{frame}{smount()}
    \insertcode{source/sys3.c}{198}{6000}{103}{121}
\end{frame}

\begin{frame}{smount()}
    \insertcode{source/sys3.c}{198}{6000}{122}{139}
\end{frame}


%-------------------------------------------------------------------------------


\subsection{iget()}

\begin{frame}{iget()}
    \begin{itemize}
        \item Suche eines Inodes: Device"=ID + Inode"=ID
        \item Berücksichtigt Locking

        \bigskip

        \item Zuerst Suche im Speicher (Inode"=Tabelle)
        \item Ansonsten einlesen und in Tabelle speichern
    \end{itemize}
\end{frame}

\begin{frame}{iget()}
    \begin{algorithmic}[1]
        \For{jeden Eintrag in Inode-Tabelle}
            \If{schon drin}
                \If{gelockt}
                    \State warten, von vorne anfangen
                \EndIf
                \If{gemountet}
                    \State Mount-Tabelle durchsuchen
                    \State Neustart mit gemountetem Dateisystem
                \EndIf
                \State lock \& return
            \EndIf
            \State erste freie Lücke in Tabelle merken
        \EndFor
        \State Inode in Tabelle initialisieren
        \State Inode von Disk lesen und Daten kopieren
    \end{algorithmic}
\end{frame}

\begin{frame}{iget()}
    \insertcode{source/iget.c}{243}{7250}{27}{32}
\end{frame}

\begin{frame}{iget()}
    \insertcode{source/iget.c}{243}{7250}{34}{59}
\end{frame}

\begin{frame}{iget()}
    \insertcode{source/iget.c}{243}{7250}{60}{69}
\end{frame}

\begin{frame}<1>[label=iget]{iget()}
    \insertcode{source/iget.c}{243}{7250}{70}{85}
\end{frame}

\againframe<2>{disklayout}

\begin{frame}{ino.h}
    \lstinputlisting[lastline=17]{source/ino.h}
    \begin{multline*}
        \texttt{int}
        +
        \texttt{char}
        +
        \texttt{char}
        +
        \texttt{char}
        +
        \texttt{char}
        \\
        +
        \texttt{char*}
        +
        \texttt{int[8]}
        +
        \texttt{int[2]}
        +
        \texttt{int[2]}
        = 32\,\text{Byte}
    \end{multline*}
\end{frame}

\againframe<2>{iget}


%-------------------------------------------------------------------------------


\subsection{getfs()}

\begin{frame}{getfs()}
    \begin{itemize}
        \item Sucht Superblock für übergebene Device"=ID in mount Tabelle

        \bigskip

        \item Überprüft einige Werte auf mögliche Fehler
            \begin{itemize}
                \item Größe der Inode"=Liste
                \item Größe der Liste freier Blöcke
            \end{itemize}
    \end{itemize}
\end{frame}

\begin{frame}{getfs()}
    \insertcode{source/alloc.c}{334}{6900}{268}{286}
\end{frame}


%-------------------------------------------------------------------------------


\subsection{update()}

\begin{frame}{update()}
    \begin{itemize}
        \item Entspricht \texttt{sync}
        \item Wird regelmäßig (alle 30 s) aufgerufen

        \bigskip

        \item Schreibt modifizierte Super Blocks
        \item Schreibt modifizierte Inodes
        \item Erzwingt Schreiben verzögerter Blöcke
    \end{itemize}
\end{frame}

\begin{frame}{update()}
    \insertcode{source/alloc.c}{334}{6900}{302}{310}
\end{frame}

\begin{frame}{update()}
    \insertcode{source/alloc.c}{334}{6900}{311}{323}
\end{frame}

\begin{frame}{update()}
    \insertcode{source/alloc.c}{334}{6900}{324}{329}
\end{frame}

\begin{frame}{update()}
    \insertcode{source/alloc.c}{334}{6900}{330}{332}
\end{frame}


%-------------------------------------------------------------------------------


\subsection{sumount()}

\begin{frame}{sumount()}
    \begin{itemize}
        \item Unmounten eines Dateisystems
        \medskip
        \item Fehler, falls noch eine Datei geöffnet ist
    \end{itemize}
\end{frame}

\begin{frame}{sumount()}
    \insertcode{source/sys3.c}{198}{6000}{145}{159}
\end{frame}

\begin{frame}{sumount()}
    \insertcode{source/sys3.c}{198}{6000}{161}{174}
\end{frame}


%-------------------------------------------------------------------------------


\subsection{alloc()}

\begin{frame}{alloc()}
    \begin{itemize}
        \item Sucht einen freien Block auf einem Datenträger
    \end{itemize}
\end{frame}

\begin{frame}{Liste freier Blöcke}
    \begin{itemize}
        \item Bis zu 100 freie Blöcke im Hauptspeicher
        \item 0. Eintrag verlinkt Block mit freien Blöcken
        \item Allererster Eintrag ist 0: Keine weiteren freien Blöcke
        \item FILO
    \end{itemize}

    \centering
    \begin{tikzpicture}[start chain=going right, >=stealth', node distance=7mm]
        \node [on chain] (a) {
            \begin{tabu}{|@{\,}c@{\,}|c|}
                \hline
                0 & 0 \\
                \hline
                1 & \qquad\qquad \\
                \hline
                2 & \\
                \hline
                \multicolumn{2}{|c|}{\vdots} \\
                \hline
                98 & \\
                \hline
                99 & \\
                \hline
            \end{tabu}
        };
        \node [on chain] (b) {
            \begin{tabu}{|@{\,}c@{\,}|c|}
                \hline
                0 & \\
                \hline
                1 & \qquad\qquad \\
                \hline
                2 & \\
                \hline
                \multicolumn{2}{|c|}{\vdots} \\
                \hline
                98 & \\
                \hline
                99 & \\
                \hline
            \end{tabu}
        };
        \node [on chain] (c) {
            \begin{tabu}{|@{\,}c@{\,}|c|}
                \hline
                0 & \\
                \hline
                1 & \qquad\qquad \\
                \hline
                2 & \\
                \hline
                \multicolumn{2}{|c|}{\vdots} \\
                \hline
                98 & \\
                \hline
                99 & \\
                \hline
            \end{tabu}
        };
        \node [on chain, label=below:\texttt{s\_free[100]}] (d) {
            \begin{tabu}{|@{\,}c@{\,}|c|}
                \hline
                0 & \\
                \hline
                1 & \qquad\qquad \\
                \hline
                2 & \\
                \hline
                \multicolumn{2}{|c|}{\vdots} \\
                \hline
                98 & \\
                \hline
                99 & \\
                \hline
            \end{tabu}
        };

        \draw [->] (b.127) to [out=180, in=90] (a.north);
        \draw [->, dotted] (c.127) to [out=180, in=90] (b.north);
        \draw [->] (d.127) to [out=180, in=90] (c.north);
    \end{tikzpicture}
\end{frame}

\begin{frame}{alloc()}
    \insertcode{source/alloc.c}{334}{6900}{57}{71}
\end{frame}

\begin{frame}{alloc()}
    \insertcode{source/alloc.c}{334}{6900}{72}{85}
\end{frame}

\begin{frame}{alloc()}
    \insertcode{source/alloc.c}{334}{6900}{87}{92}
\end{frame}


%-------------------------------------------------------------------------------


\subsection{itrunc()}

\begin{frame}{itrunc()}
    \begin{itemize}
        \item Gibt alle in einem Inode referenzierte Blöcke frei
        \medskip
        \item Rückwärts um Fragmentierung entgegenzuwirken
    \end{itemize}
\end{frame}

\begin{frame}{itrunc()}
    \begin{algorithmic}[1]
        \If{Special File} \Return \EndIf
        \For{jeden Zeiger auf Block in Inode rückwärts}
            \If{indirekte Addressierung}
                \State Lese Block
                \For{alle Einträge rückwärts}
                    \If{ist letzter Eintrag in Inode}
                        \Comment Doppelt indirekt
                        \State Lese Block
                        \For{alle Einträge rückwärts}
                            \State Lösche Block
                        \EndFor
                    \EndIf
                    \State Lösche Block
                \EndFor
            \EndIf
            \State Lösche Block
        \EndFor
        \State Inode zurücksetzen
    \end{algorithmic}
\end{frame}

\begin{frame}{itrunc()}
    \insertcode{source/iget.c}{243}{7250}{165}{173}
\end{frame}

\begin{frame}{itrunc()}
    \insertcode{source/iget.c}{243}{7250}{174}{195}
\end{frame}

\begin{frame}{itrunc()}
    \insertcode{source/iget.c}{243}{7250}{196}{200}
\end{frame}


%-------------------------------------------------------------------------------


\subsection{free()}

\begin{frame}{free()}
    \begin{itemize}
        \item Fügt einen Block in die Liste freier Blöcke ein
        \medskip
        \item Wenn List voll, schreibe sie auf den Datenträger
    \end{itemize}
\end{frame}

\begin{frame}{free()}
    \insertcode{source/alloc.c}{334}{6900}{101}{114}
\end{frame}

\begin{frame}{free()}
    \insertcode{source/alloc.c}{334}{6900}{115}{128}
\end{frame}


%-------------------------------------------------------------------------------


\subsection{iput()}

\begin{frame}{iput()}
    \begin{itemize}
        \item Dekrementiert Referenz"=Zähler eines Inodes
        \item Gibt Inode im Hauptspeicher frei, falls letzte Referenz
        \item Löscht Datei, falls kein Link mehr
    \end{itemize}
\end{frame}

\begin{frame}{iput()}
    \insertcode{source/iget.c}{243}{7250}{95}{115}
\end{frame}


%-------------------------------------------------------------------------------


\subsection{ifree()}

\begin{frame}{ifree()}
    \begin{itemize}
        \item Markiert Inode als frei\\
            Also: In \texttt{s\_inode} Liste des Super Blocks einfügen
        \medskip
        \item Inode wird verworfen, falls Liste voll ist
    \end{itemize}
\end{frame}

\begin{frame}{ifree()}
    \insertcode{source/alloc.c}{334}{6900}{235}{246}
\end{frame}


%-------------------------------------------------------------------------------


\subsection{iupdat()}

\begin{frame}{iupdat()}
    \begin{itemize}
        \item Aktualisiert Inode auf Datenträger
    \end{itemize}
\end{frame}

\begin{frame}{iupdat()}
    \insertcode{source/iget.c}{243}{7250}{125}{132}
\end{frame}

\begin{frame}{iupdat()}
    \insertcode{source/iget.c}{243}{7250}{133}{153}
\end{frame}


\end{document}
